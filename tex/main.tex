% ****** Start of file apssamp.tex ******
%
%   This file is part of the APS files in the REVTeX 4.2 distribution.
%   Version 4.2a of REVTeX, December 2014
%
%   Copyright (c) 2014 The American Physical Society.
%
%   See the REVTeX 4 README file for restrictions and more information.
%
% TeX'ing this file requires that you have AMS-LaTeX 2.0 installed
% as well as the rest of the prerequisites for REVTeX 4.2
%
% See the REVTeX 4 README file
% It also requires running BibTeX. The commands are as follows:
%
%  1)  latex apssamp.tex
%  2)  bibtex apssamp
%  3)  latex apssamp.tex
%  4)  latex apssamp.tex
%
\documentclass[%
 reprint,
%superscriptaddress,
%groupedaddress,
%unsortedaddress,
%runinaddress,
%frontmatterverbose, 
%preprint,
%preprintnumbers,
%nofootinbib,
%nobibnotes,
%bibnotes,
 amsmath,amssymb,
 aps,
%pra,
%prb,
%rmp,
%prstab,
%prstper,
%floatfix,
]{revtex4-2}

\usepackage{graphicx}% Include figure files
\usepackage{dcolumn}% Align table columns on decimal point
\usepackage{bm}% bold math
%\usepackage{hyperref}% add hypertext capabilities
%\usepackage[mathlines]{lineno}% Enable numbering of text and display math
%\linenumbers\relax % Commence numbering lines

%\usepackage[showframe,%Uncomment any one of the following lines to test 
%%scale=0.7, marginratio={1:1, 2:3}, ignoreall,% default settings
%%text={7in,10in},centering,
%%margin=1.5in,
%%total={6.5in,8.75in}, top=1.2in, left=0.9in, includefoot,
%%height=10in,a5paper,hmargin={3cm,0.8in},
%]{geometry}

\begin{document}

\preprint{APS/123-QED}

%\title{Manuscript Title:\\with Forced Linebreak}% Force line breaks with \\
\title{Cosmology with standard sirens}
%\thanks{A footnote to the article title}%

\author{Bernardo Porto Veronese}
%\altaffiliation[Also at ]{PPGCOSMO, UFES}%Lines break automatically or can be forced with \\
\email{Second.Author@institution.edu}
\affiliation{%
	PPGCOSMO, UFES
}%

%\collaboration{MUSO Collaboration}%\noaffiliation

\date{\today}% It is always \today, today,
%  but any date may be explicitly specified

\begin{abstract}
	An article usually includes an abstract, a concise summary of the work
	covered at length in the main body of the article.
	\begin{description}
		\item[Usage] Secondary publications and information retrieval purposes.
		\item[Structure] You may use the \texttt{description} environment to structure your abstract; use the
		      optional argument of the \verb+\item+ command to give the category of each item.
	\end{description}
\end{abstract}

%\keywords{Suggested keywords}%Use showkeys class option if keyword
%display desired
\maketitle

%\tableofcontents

\section{\label{sec:introduction}Introduction} The idea of using gravitational waves (GWs) from compact binary mergers to measure
cosmological parameters was first introduced by Bernard Schutz in 1986~\cite{Schutz:1986gp}. for
cosmology is an idea which has finally come to fruition in recent years. These signals directly
provide a measurement of the luminosity distance measurement to the source, which is therefore
independent of the cosmic distance ladder. With the addition of redshift information, measurements
can therefore be made of those cosmological parameters which impact the expansion history of the
Universe, such as the Hubble constant ($H_0$). This approach is independent of all other local
measurements to date.

The standard siren method probes the expansion history of the universe with the distance-redshift
relation, with which one can infer the cosmological parameters such as $H_0$ and the dark energy
equation of state parameter $w$:~\cite{Hogg:1999ad}

\begin{align}
	\label{eq:intro:distance-redshift-relation}
	D_l(z)           & = (1+z)\frac{c}{H_0 \sqrt{\Omega_{K}}} \sinh \left[ \sqrt{\Omega_{K}} \int_0^z \frac{H_0}{H(z') dz'}\right] \\
	\nonumber
	\frac{H(z)}{H_0} & = \sqrt{\Omega_m (1+z)^3 + \Omega_K (1+z)^2 + \Omega_{de} (1+z)^{3(1+w)}}.
\end{align}

To lighten notation, we have omitted the 0-subscript next to the $\Omega_i$'s, although they
correspond to the present day values in the above equation. Note that
using~\eqref{eq:intro:distance-redshift-relation} requires specifying a cosmological model.

\subsection{Gravitational-wave distances}

The accuracy of the GW luminosity distance measurement is typically of the order of 10\%. The main
source of uncertainty comes from the degeneracy between the distance and inclination angle of the
source. The latter is defined as the angle between the line-of-sight vector from the source to the
detector and the orbital-angular momentum of the binary system.

\subsection{Assigning redshifts to GW sources}

From the GW data, it is possible to infer the luminosity distance to the binary source, but not the
redshift, as the latter comes degenerate with the chirp mass in the GW waveform modelling. It is
therefore necessary to complement it with another source of information that provides the redshift
measurement. If an electromagnetic (EM) counterpart to the gravitational wave event is detected,
then one can identify the host galaxy and, therefore, the redshift of the source. An event with an
EM counterpart is called a \textit{bright siren}. So far, the only confirmed such event has been
the binary neutron star detection GW170817, which occurred so exceptionally close to our galaxy -
at $d \sim 40$Mpc - that a direct, model-independent estimation of $H_0$ with Hubble's law,

\begin{equation}
	v_H = H_0 d,
\end{equation}

could be made by measuring the Hubble flow velocity $v_H$, resulting in $H_0 = 70.0^{+12.0}_{-8.0}$
km s$^{-1}$ Mpc$^{-1}$~\cite{LIGOScientific:2017adf}.

As stated above, almost all GW events have been detected without an EM counterpart. These
\textit{dark sirens} can be used to probe the expansion of the universe provided that they are
complemented with an external redshift measurement. In his original paper, Schutz suggested that
galaxy catalogs could be such a source: an averaging redshifts of the galaxies within the GW
event's localisation volume could serve as an estimate of true value of $z$. Such analyses have
been carried out in the literature, see~\onlinecite{}

\section{\label{sec:framework}Statistical framework}

\bibliography{refs}% Produces the bibliography via BibTeX.

\end{document}
%
% ****** End of file apssamp.tex ******
